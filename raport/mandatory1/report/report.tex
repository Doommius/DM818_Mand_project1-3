%Template by Mark Jervelund - 2015 - mjerv15@student.sdu.dk

\documentclass[a4paper,10pt,titlepage]{report}

\usepackage[utf8]{inputenc}
\usepackage[T1]{fontenc}
\usepackage[english]{babel}
\usepackage{amssymb}
\usepackage{amsmath}
\usepackage{amsthm}
\usepackage{graphicx}
\usepackage{fancyhdr}
\usepackage{lastpage}
\usepackage{listings}
\usepackage{algorithm}
\usepackage{algpseudocode}
\usepackage[document]{ragged2e}
\usepackage[margin=1in]{geometry}
\usepackage{enumitem}
\usepackage{color}
\usepackage{datenumber}
\usepackage{venndiagram}
\usepackage{chngcntr}
\setdatetoday
\addtocounter{datenumber}{0} %date for dilierry standard is today
\setdatebynumber{\thedatenumber}
\date{}
\setcounter{secnumdepth}{0}
\pagestyle{fancy}
\fancyhf{}

\newcommand{\Z}{\mathbb{Z}}
\lhead{Computer Architecture (DM548))}
\rhead{Mark Jervelund (Mjerv15)}
\rfoot{Page  \thepage \, of \pageref{LastPage}}
\counterwithin*{equation}{section}

\begin{document}
\renewcommand{\thepage}{\roman{page}}% Roman numerals for page counter
\tableofcontents
\newpage
\setcounter{page}{1}
\renewcommand{\thepage}{\arabic{page}}
\section{Project description}
To describe the services an operating system provides to
users, processes, and other systems\\
To discuss the various ways of structuring an operating
system\\
To explain how operating systems are installed and
customized and how they boot\\
\newpage

\section{Introduction}



\section{Design}






\section{appendix}
\begin{lstlisting}
 https://software.intel.com/sites/landingpage/IntrinsicsGuide/#techs=AVX_512&cats=Elementary%20Math%20Functions,Load 
	
https://www.codeproject.com/Articles/874396/Crunching-Numbers-with-AVX-and-AVX

_mm<bit_width>_<name>_<data_type> 

The parts of this format are given as follows:

<bit_width> identifies the size of the vector returned by the function. For 128-bit vectors, this is empty. For 256-bit vectors, this is set to 256.

<name> describes the operation performed by the intrinsic
<data_type> identifies the data type of the function's primary arguments


Data Type	Description
__m128i	128-bit vector containing 8 integers
__m128	128-bit vector containing 4 floats
__m128d	128-bit vector containing 2 doubles
__m256i	256-bit vector containing 16 integers
__m256	256-bit vector containing 8 floats
__m256d	256-bit vector containing 4 doubles
__m512i	512-bit vector containing 32 integers
__m512	512-bit vector containing 16 floats
__m512d	512-bit vector containing 8 doubles



ps - vectors contain floats (ps stands for packed single-precision)
pd - vectors contain doubles (pd stands for packed double-precision)
epi8/epi16/epi32/epi64 - vectors contain 8-bit/16-bit/32-bit/64-bit signed integers
epu8/epu16/epu32/epu64 - vectors contain 8-bit/16-bit/32-bit/64-bit unsigned integers
si128/si256 - unspecified 128-bit vector or 256-bit vector
m128/m128i/m128d/m256/m256i/m256d - identifies input vector types when they're different than the type of the returned vector

\end{lstlisting}

\section{notes from lecture with Jakob}

Our cache use, calculate it to be = to 256 or maybe a bit less.
$kc and mc should be large. n_c is less important.$
kc = 128 MC = 128, 
$8* (m_c*k_c+m_c*n_c+N_c+k_c)$


$avx_256 registers are called ymm0 to 15$
$avx_512 registers are called zmm0 to 31$


Make function for each size of slices..
128 = 4.4 
256 = 8.4 
512 = 8.8




































\end{document}
